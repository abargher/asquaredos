\documentclass[12pt]{article}
\usepackage[colorlinks,allcolors=blue]{hyperref}
% \usepackage{amsmath,amsfonts,graphicx,amssymb}
% \usepackage{latexsym}
% \usepackage{subfig}
% \usepackage{svg}
% \usepackage{enumitem}
% \usepackage{adjmulticol}
% \usepackage{listings}
\usepackage{fullpage}

\newcommand\fnsep{\textsuperscript{,}}
\newcommand{\os}{\texttt{A$^2$OS} }
\newcommand{\osns}{\texttt{A$^2$OS}}

\title{\texttt{A$^2$OS}: a minimal ``OS'' for the Raspberry Pi Pico}
\author{Alec Bargher and Gus Waldspurger}
\date{}

\begin{document}
\maketitle

\section{Introduction}
\os is a minimal ``operating system'' for the RP2040 microcontroller, providing
context switching and process scheduling, enabling seamless multiprogramming.
Although the system is implemented using a Raspberry Pi Pico, we hope that \os
will be compatible with any application of the RP2040 chip with minimal
modification.

\vspace{.5em}

\os implements a minimal kernel, responsible for creating, destroying, and
scheduling ``user'' processes. Basic utilities are also provided for these
processes, in the form of system calls to acquire and release memory. As the
kernel performs no background tasks, its code is only executed asynchronously
after the initial boot sequence is completed, whenever users or system resources
trigger an interrupt. Users will load the kernel onto the board once, and may
then flash an arbitrary set of user programs specifically compiled and linked
for compatibility with \os.

\section{Motivation}

\os is largely motivated by the desire to share CPU time on an ARM Cortex M0+
processor. This architecture---of which the RP2040 is notable example---offer a
powerful compute platform while maintaining extremely low cost and power
requirements. Processors like the RP2040 are thus desirable for a variety of
complex embedded applications. However, properly taking advantage of the
power these processors offer can be challenging and impractical due to the
burden of CPU management being placed entirely on the programmer. \os hopes to
solve this issue.

With \osns, CPU management is abstracted away; the programmer is now responsible
only for the sharing and synchronization of other, more diverse system
resources (e.g., I/O).

\section{Existing Work \& References}

The problem of context switching and process/task scheduling for
microcontrollers is not a new one, and there are a variety of existing solutions
for this problem. For example,
\href{https://www.freertos.org/Documentation/02-Kernel/02-Kernel-features/01-Tasks-and-co-routines/00-Tasks-and-co-routines#characteristics-of-a-task}{FreeRTOS}
provides task scheduling, context switching, and dynamic memory allocation, but
also includes a plethora of other features, including a TCP stack, stream
buffers for inter-task communication, and synchronization primitives like
semaphores and locks. Although in some use cases these features will be useful,
we see the simplicity and limited feature set of \os as a benefit. This allows
us to minimize memory and processing overhead for the kernel, reserving the vast
majority of system resources for user processes. More specialized
microcontroller operating systems have also implemented multitasking, including
\href{https://www.youtube.com/watch?v=yHqaspeGJRw}{Chromium-EC}\footnote{We
referenced this talk and its corresponding slides while implementing context
switching.}.

Unlike these existing solutions, \os aims to provide a general and freely
available solution to multitasking by providing only the minimal facilities
required to effectively share CPU time and main memory, rather than implementing
an entire Real Time OS (RTOS).


\section{System Design}
\subsection{Kernel layout}
In designing \os, we prioritized reducing complexity, while also handling the
constraints of the RP2040 platform---the largest of which is the lack of
hardware support for virtual memory. Virtual memory is uncommon for
microcontrollers, so embedded programs typically deal with hardware memory
addresses directly. Lacking the isolation afforded by virtual memory, we chose
to place the kernel at the top of physical memory, with fixed memory addresses
for its critical global variables and structures, and a fixed-size stack for its
few function calls. By fixing the size and location of the kernel, the memory
protection unit (MPU) can be used to minimally isolate kernel state from
userspace programs. The kernel maintains structures for tracking
\begin{itemize}
    \item {A pointer to the process control block (PCB) of the currently active
    process,}
    \item A pointer to the scheduler's ready queue,
    \item A pointer to the heap allocator's free blocks list,
    \item Memory zones and associated metadata for use by the zone allocator,
\end{itemize}
among the other state the kernel must maintain.
%% Continue with more on kernel layout here...

\subsection{Memory Allocation}
The Pi Pico, like many other microcontrollers, maps its peripherals and other
hardware components to memory addresses in an extended address space which
exceeds the total size of physical memory. Hardware peripherals accessed by
their memory-mapped addresses are not currently managed by \osns.

The kernel's state is largely organized as a series of zones, which are managed
by a zone allocator. The zone allocator is used only by the kernel to allocate
fixed-size elements from these pre-allocated zones of kernel memory, organized
by element type. This system is currently only used to maintain PCBs, but is
general and may be trivially expanded with additional zones.

All physical memory (SRAM) not occupied by the kernel is managed by a
kernel-private heap allocator, which is utilized whenever process-owned memory
must be allocated. For example, the heap is used to allocate each process's
fixed size (4KB) stack at the time of that process's creation. Note that code
sections of a process are stored in executable flash (XIP) storage, rather than
in memory. Processes may also explicitly request and release memory via system
call, which is ultimately managed via the same kernel-private heap. The
implementation of an efficient userspace heap is left as a responsibility for
the programmer.\footnote{Note
that the kernel's heap allocator is (currently) an extremely simple one. It
makes no attempts to reduce fragmentation, and will choose the first available
block on the heap that can fit the requested memory.}\fnsep
\footnote{Security-minded readers may also notice that because the kernel doles
out memory from the entirety of SRAM (except the kernel's private state), there
is no isolation of any in-memory process data between processes. While this is
unfortunate, it is ultimately unavoidable due to the MPU's limit of protecting
just 8 regions---insufficient to provide proper protection to each block of
memory allocated by the kernel for a user process. Handling this lack of
isolation is thus made the responsibility of the programmer and system
operator.}

%% Add more miscellaneous elements here, or add more sections above this line.
\subsection{Other design elements}
As long as users are careful to only flash programs set up properly for \osns,
then the kernel will not need to be re-flashed. This feature is provided by
\osns's boot sequence and fixed kernel location and layout. Since \os user
programs are never saved to the same location as the kernel and the flash memory
is not fully cleared, user programs will not overwrite the kernel when flashed
correctly.

Since the kernel implements a ``third stage'' bootloader that loads the kernel
and user programs into their correct memory locations without relying on the
standard RP2040 bootloader, the kernel can return to its initial first-boot
state when recovering from a reset or power loss.

\end{document}
