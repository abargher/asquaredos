\documentclass[12pt]{article}
\usepackage[colorlinks,allcolors=blue]{hyperref}
% \usepackage{amsmath,amsfonts,graphicx,amssymb}
% \usepackage{latexsym}
% \usepackage{subfig}
% \usepackage{svg}
% \usepackage{enumitem}
% \usepackage{adjmulticol}
% \usepackage{listings}
\usepackage[
  margin=2.5cm,
  includefoot,
  footskip=30pt,
]{geometry}

\newcommand\fnsep{\textsuperscript{,}}
\newcommand{\os}{\texttt{A$^2$OS} }
\newcommand{\osns}{\texttt{A$^2$OS}}

\title{\texttt{A$^2$OS}: a minimal OS for the Raspberry Pi Pico}
\author{Alec Bargher and Gus Waldspurger}
\date{}

\begin{document}
\maketitle

\section{Introduction}
\os is a minimal operating system for the RP2040 microcontroller, providing
context switching, preemptive process scheduling, and virtual memory, enabling
seamless multiprogramming. Although the system was developed on a Raspberry Pi
Pico, we believe that \os can, with minimal modifications, be compatible with
most RP2040-based boards.

\vspace{.5em}

\os implements a minimal kernel, responsible for creating, destroying, and
scheduling user processes. The kernel performs no background tasks, and thus
after the initial boot sequence it executes only asynchronously when interrupt
handlers are invoked. Users load the kernel onto the board once, and may then
flash an arbitrary set of user programs specifically compiled and linked for
compatibility with \os.

\section{Motivation}

\os is largely motivated by the desire to timeshare an ARM Cortex M0+ processor.
This architecture---of which the RP2040 is notable example---offer a powerful
compute platform while maintaining extremely low cost and power requirements.
Microcontrollers like the RP2040 are thus desirable for a variety of complex
embedded applications. However, properly taking advantage of the resources these
processors offer can be challenging and often impractical due to the burden of
CPU and memory management being placed entirely on the programmer. \os hopes to
solve this issue.

With \osns, CPU management is abstracted away; the programmer is now responsible
only for the sharing and synchronization of other, more diverse system
resources (e.g., I/O). Furthermose, \os provides a well-featured virtual memory
system with address space isolation and demand paging, abstracting away the need
consider how memory need be laid out and accessed by multiple independent
processes.

\section{Existing Work \& References}

The problem of context switching and process/task scheduling for
microcontrollers is not a new one, and there are a variety of existing solutions
for this problem. For example,
\href{https://www.freertos.org/Documentation/02-Kernel/02-Kernel-features/01-Tasks-and-co-routines/00-Tasks-and-co-routines#characteristics-of-a-task}{FreeRTOS}
provides task scheduling, context switching, and dynamic memory allocation, but
also includes a plethora of other features, including a TCP stack, stream
buffers for inter-task communication, and synchronization primitives like
semaphores and locks. Although in some use cases these features will be useful,
we see the simplicity and limited feature set of \os as a benefit. This allows
us to minimize memory and processing overhead for the kernel, reserving the vast
majority of system resources for user processes. More specialized
microcontroller operating systems have also implemented multitasking, including
\href{https://www.youtube.com/watch?v=yHqaspeGJRw}{Chromium-EC}\footnote{We
referenced this talk and its corresponding slides while implementing context
switching.}.

Unlike these existing solutions, \os aims to provide a general and freely
available solution to multitasking by providing only the minimal facilities
required to effectively share CPU time and main memory, rather than implementing
an entire Real Time OS (RTOS). \os also differentiates itself from these other
solutions through its optional virtual memory system. No other systems provide
this feature directly due to the lack of hardware support, but by using and 
modifying the
\href{https://dmitry.gr/?r=05.Projects&proj=27.%20m0FaultDispatch}{\texttt{m0FaultDispatch} project}
by Dmitry Grinberg, we could infer extra information about the cause of a
HardFault, allowing us to implement a virtual memory system.


\section{System Design}
\subsection{Kernel layout}
In designing \os, we prioritized reducing complexity, while also handling the
constraints of the RP2040 platform. We place the kernel at the top of physical
memory, with fixed memory addresses for its critical global variables and
structures. Because the RP2040 includes two regions of memory---4 banks of 64KB
SRAM which are striped to for a 256KB region, and also 2 banks of 4KB at the
top of the SRAM address range---we restrict user programs to accessing the 256KB
of striped SRAM not occupied by kernel state, and then leave the 4KB scratch
banks for the kernel's stack. Kernel state includes:
\begin{itemize}
    \item {A pointer to the process control block (PCB) of the currently active
    process,}
    \item A pointer to the scheduler's ready queue,
    \item Memory zones and associated metadata for use by the zone allocator,
    \item {A 64KB "write cache" used as an additional memory tier that allows
    the most frequently dirtied pages to remain in SRAM, rather than to flash,
    thus reducing the number of flash write-erase cycles}
\end{itemize}
In sum, the kernel consumes roughly 72KB of the 256KB of the available striped
SRAM. The \emph{large} size of the write cache is necesitated by the limited
number of protection regions offered by the MPU. Only 8 regions exist, and 2
regions must be used as "background" regions to ensure the correct permissions
are used for non-SRAM addresses, and when an MPU subregion is disabled. This
leaves just 6 configurable regions for the VM system, and because regions' sizes
must be a power of 2, we decided to use 32KB regions, allowing up to 192KB of
striped SRAM to be managed by the VM system. Thus, as 64KB would be unusable by
user programs, we decided to make a very large write cache, as its size directly
correlates to reduced flash wear.
%% Continue with more on kernel layout here...

\subsection{Memory Allocation}
The Pi Pico, like many other microcontrollers, maps its peripherals and other
hardware components to memory addresses in an extended address space which
exceeds the total size of physical memory. Hardware peripherals accessed by
their memory-mapped addresses are not currently managed by \osns.

The kernel's state is largely organized as a series of zones, which are managed
by a zone allocator. The zone allocator is used only by the kernel to allocate
fixed-size elements from these pre-allocated zones of kernel memory, organized
by element type. This system is currently only used to maintain PCBs, but is
general and may be trivially expanded with additional zones.

All physical memory (SRAM) not occupied by the kernel is managed by a
kernel-private heap allocator, which is utilized whenever process-owned memory
must be allocated. For example, the heap is used to allocate each process's
fixed size (4KB) stack at the time of that process's creation. The
implementation of an efficient userspace heap is left as a responsibility for
the programmer, if desired.\footnote{Note
that the kernel's heap allocator is (currently) an extremely simple one. It
makes no attempts to reduce fragmentation, and will choose the first available
block on the heap that can fit the requested memory.}

%% Add more miscellaneous elements here, or add more sections above this line.
\subsection{Other design elements}
As long as users are careful to only flash programs set up properly for \osns,
then the kernel will not need to be re-flashed. This feature is provided by
\osns's boot sequence and fixed kernel location and layout. Since \os user
programs are never saved to the same location as the kernel and the flash memory
is not fully cleared, user programs will not overwrite the kernel when flashed
correctly.

Since the kernel implements a ``third stage'' bootloader that loads the kernel
and user programs into their correct memory locations without relying on the
standard RP2040 bootloader, the kernel can return to its initial first-boot
state when recovering from a reset or power loss.

\section{Implementation progress}

We successfully implemented both multiprocessing with preemptive scheduling, and
a surprisingly well-featured (given hardware limitations) virtual memory system
which provides both address space isolation and demand paging for user programs.

The biggest change from the originally proposed design is the inclusion of a
virtual memory system. Although the RP2040 does not have a memory management
unit (MMU) to properly support address translation and paging, we were able
to co-opt the \emph{memory protection unit} (MPU) to emulate the behavior of a
page fault. We use the MPU, together with a custom HardFault handler, in order
to deduce the cause of a HardFault, and initiate our custom page fault handler
for relevant MPU-related faults.

Using \os we were able to run two ordinary, unmodified user programs, which were
each able to independently control peripherals via GPIO pins. Both programs'
binaries were linked to begin from the \emph{same address}, and our VM system
successfully paged in the contents of each program's respective binary from
flash on demand.

\textbf{NOTE}: All of the text in this section and any following sections can
and should probably be changed-- this is a general idea outline for now.

\textbf{TODO}: include the following expansions on the content:

\begin{itemize}
    \item discussions of limitations and related works
\end{itemize}

\textbf{TODO}: Integrate changes into the previous sections and cut out content
from proposal that is no longer relevant or correct.
\vspace{1em}



\section{Design evolution}
Much of the general design has remained from the original proposal, however, a
few critical elements have changed in the current version of the system.

\subsection{Process address space isolation}
The introduction of the virtual memory system allows us to properly isolate
the address spaces of user programs from the kernel, and from and each other.
When VM is enabled, user programs can be linked to arbitrary (potentially
identical or overlapping) addresses. Since code and data are paged in lazily
upon context switch to a new process, user programs can never access the
contents of another user program's address space. MPU protections are reset on
every context switch so that the current process's content is freshly paged in
on each access. The MPU also protects the kernel's state (which is maintained in
a disjoint region of memory from process address spaces, and is not managed by
the VM system) so that it is only accessible from an interrupt context and never
from standard user program mode.

\subsection{Userprogram memory allocation}
With the introduction of the virtual memory system, there is no longer any need
for a system call providing heap memory allocation to user programs. Now, all
user programs are able to access the entire memory address space, as if there
were no OS at all, so the virtual memory system will simply page in a 
zero-filled page whenever a new region of the address space is accessed for the
first time.

\section{Current limitations}

\subsection{Limitations of address space isolation}
Due to the limited number of configurable (sub)regions provided by the MPU, the
virtual memory system is only able to properly isolate the portion of the
address space covering SRAM. Flash and other memory-mapped devices are not
protected by the MPU, and it is thus the responsibility of user programs to not
access the same shared resources. This limitation could be addressed by
preventing all userspace accesses outside of SRAM, and implementing system calls
to manage hardware components such as flash storage and other peripherals. We
chose not to pursure this approach due to the significant performance cost which
would be incurred by user programs.

\end{document}
